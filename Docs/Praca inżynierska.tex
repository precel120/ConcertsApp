\documentclass[12pt]{article}
\usepackage[utf8]{inputenc}
\usepackage{color}
\usepackage{polski}
\usepackage{graphicx}
\usepackage{pdfpages}
\usepackage{indentfirst}
\usepackage{geometry}
\usepackage{setspace}
\usepackage[hyphens]{url}

\setstretch{1.5}

 \geometry{
 a4paper,
 left=30mm,
 top=25mm,
 right=10mm,
 bottom=20mm
 }

\begin{document}
\begin{sloppypar}
\includepdf{strona_tytulowa.pdf}

\tableofcontents
\newpage

\section{Wprowadzenie}
{
  W dobie trwającego rozwoju technologicznego, każdy aspekt życia codziennego jest usprawniany i przenoszony
  do internetu. Nieinaczej jest z promowaniem i sprzedażą biletów na wydarzenie kulturowe pokroju:
  spektakli w teatrze, koncertów, festiwali. Aplikacja ConcertsApp ma służyć dokładnie temu,
  ma zminimalizować czas, jaki należałoby kiedyś przeznaczyć na zdobycie biletów. Czy też na uzyskanie
  informacji o wydarzeniach z interesującej konsumenta dziedziny, lub odbywającę się w pobliżu.
\subsection{Problematyka}
{
  
}
\subsection{Cel i założenia pracy}
{
  Celem niniejszej pracy dyplomowej jest stworzenie aplikacji webowej, która umożliwi w łatwy i szybki
  sposób na promowanie wydarzeń kulturowych i zakup biletów na nie. Zakup odbywać się będzie za pomocą
  płatności online, jedynie podając dane karty kredytowej lub debetowej.\par
  Zakres pracy obejmuję zaprojektowanie i implementację aplikacji klienckiej i serwerowej, przy
  wykorzystaniu, najnowszych i bardzo popularnych na rynku pracy, technologii. Do strony klienckiej
  został wykorzystany React, z koleji do stworzenia serwera użyto Node.JS.
}
\subsection{Struktura pracy inżynierskiej}
{
  Pierwszy rozdział ma posłużyć jako wprowadzenie do problemu podjętego w niniejszej pracy dyplomowej.
  Następny przedstawia aspekty takie jak: konkurencyjne rozwiązania do tworzonej aplikacji,
  analiza najpopularniejszych metod płatności online, opowiada o bazach danych oraz przybliża
  metody tworzenie aplikacji webowych.
  Trzeci rozdział przedstawia wykorzystane technologie do implementacji aplikacji oraz uzasadnia dlaczego
  akurat ona została wybrana a nie inna.
  Czwarty rozdział pokazuje jak przebiegał proces twórczy, czyli projektowanie i implementacja.
}
}

\section{Część teoretyczna}
{
\subsection{Analiza istniejących portali do promowania i dystrybucji biletów na wydarzenia kulturowe}
{
  Najpopularniejszymi portalami zbliżonymi do tworzonej aplikacji są zdecydowanie:
  \begin{itemize}
    \item Eventim
    \item GoingApp
  \end{itemize}
  Eventim jest znacznie starszym portalem, co można stwierdzić chociażby, po jego szacie graficznej,
  łatwo to zauważyć porównująć do wyżej wymienionego GoingApp. 
  W swojej ofercie ma wydarzenia pokroju: koncertów, przedstawień teatralnych, oper czy baletów.\\
  Same bilety sprzedawane są w formie papierowej, przychodzą na maila, 
  lub istnieje możliwość zakupu biletu mobilnego. Kupno może odbyć się zarówno poprzez strone internetową, 
  jak również poprzez aplikację mobilną. Płatności można dokonać za pomocą przelewu tradycyjnego, karty kredytowej, szybkiego przelewu Dotpay\\
  GoingApp charakteryzuje się bardziej nowoczesną i przejrzystą szatą graficzną w porównaniu do Eventim. 
  Oferty obu portali są bardzo do siebie zbliżone, lecz tutaj można znaleźć takie wydarzenia, jak chociażby imprezy związane z filmem czy jedzeniem.
  Jednakże nie posiada on biletów na balet lub opery.\\
  Wejściówki można, podobnie jak w Eventim nabyć poprzez ich stronę internetową lub aplikację mobilną. 
  Sam bilet jest dostępny w formie dokumentu PDF, lub kodu QR, dostęp do niego mamy zarówno poprzez maila,którego otrzymujemy zaraz po zakupie, oraz aplikację.
  Użytkownik może dokonać płatności za pomocą systemów płatniczych takich jak: payU, eCard, MasterPass oraz Paymento\textregistered \\
  Aplikacja webowa tworzona na potrzeby niniejszej pracy inżynierskiej czerpie z obu portali najlepsze cechy. 
  Z Eventim czerpie różnorodność wydarzeń, z koleji z GoingApp przejrzystość interfejsu i bilety w formie, bardzo popularnego kodu QR.
  Metody płatności zostają ograniczone do kart płatniczych: debetowej i kredytowej. Podawane są numer karty, data ważności oraz kod CVV.
}
\subsection{Zakupy internetowe}
{
  Sklepy internetowe z roku na rok rosną w siłę, przybywa ich liczba w zatrważającym tempie. 
  Aż 73\% ankietowanych w raporcie przygotowanym przez Gemius Polska\cite{gemius-report} deklaruje kupowanie online i ta forma zakupów cieszy się niezmiennie dobrym 
  wizerunkiem wśród kupujących. Brak sklepu internetowego stanowi ogromną niekorzyść dla obecnych przedsiębiorców. 
  Zakupy online charakteryzują się dużo większym wyborem produktów, łatowścią w porównywaniu ofert czy też łatowścią w znalezieniu interesujących produktów. 
  Lecz, to nie z wyżej wymienionych powodów, sklepy w sieci cieszą sie taką popularnością.
  Wśród najczęściej wymienianych powodów są: dostępność przez całą dobę (aż 82\% wybrało ten powód), brak konieczności wyjazdu do sklepu - 78\%, 
  nieograniczony czas wyboru - 72\% oraz atrakcyjniejsze ceny niż w sklepach tradycyjnych - 71\%.\\
  Niestety taka forma zakupów ma też swoje słabe strony, najpopularniejszymi
  wymienianymi, napotkanymi problemami są: wysokie koszta dostawy, długi czas oczekiwania na dostawę, irytujące reklamy produktów, wcześniej poszukiwanych.
  Również bardzo często wymienianą przeciwnością jest uszkodzona przesyłka w transporcie, wynikać to może ze źle zapakowanej i zabezpieczonej paczkim lub z winy 
  firmy kurierskiej i jej pracowników.\\
  Co jednak sprawia, że klienci decydują się na wybór danego portalu na zakupy?\\
  Najczęściej wybór sklepu pada dzięki kodom rabatowym, dopiero w dalszej kolejności padają hasła takie jak dokładne informacje o warunkach zamówienia,
  dostępne na stronie dane firmy czy przejrzysta i funkcjonalna strona internetowa. Wydawać by się mogło, że to na te kolejne cechy portali, powinno się w pierwszej kolejności
  zwracać uwagę, bo to dzięki nim w pierwszej chwili można stwierdzić czy strona jest fałszywa, czy też nie.
}
\subsection{Analiza najpopularniejszych internetowych metod płatności online} % strona 119 raportu, strona 80 wstep, strona 89 czynniki motywujace do robienia zakupow online
{
  Najpopularniejszymi metodami płatności w internecie, według raportu "E-commerce w Polsce 2020"\cite{gemius-report} są kolejno:
  \begin{itemize}
    \item szybki przelew przez serwis płatności np. payU, przelewy24
    \item przelew tradycyjny
    \item płatność kartą płatniczą przy składaniu zamówienia
    \item płatności mobilne np. BLIK
  \end{itemize}
  Nie zostały wymienione płatności pokroju wysyłki za pobraniem, płatności w sklepie przy odbiorze, ponieważ nie są to płatności realizowane online, 
  a tych dotyczy analiza przedstawiona w niniejszym rozdziale.\\
  Zdecydowanie nie powinna dziwić obecność szybkich przelewów na pierwszym miejscu tego rankingu. Aż 70\% ankietowanych odpowiedziało, 
  że choć raz korzystało z tej metody, przy robieniu zakupów przez internet. Cechują się błyskawiczynym czasem realizacji, w przeciwieństwie do tradycyjnych
  przelewów. Zaledwie 46\% ankietowanych zdecydowało się choć raz na przelew tradycyjny. Różnica jest znaczna, jednakże nie powinna ona dziwić, ponieważ 
  to oszczędność czasu, poniekąd sprawia, że klienci decydują się na zakupy online w pierwszej kolejności. Przelewy tradycyjne dodatkowo wydłużają czas realizacji 
  zamówienia, ponieważ ich przetwarzanie odbywa się jedynie w dni robocze, o wyznaczonych godzinach, różnych, w zależności od banku.\\
  Na płatność kartą decyduje się 40\% zapytanych, jest to dość zaskakujące, zważywszy na fakt, że jest to zdecydowanie jedna z najszybszych metod płatności. 
  Do jej realizacji niezbędne jest jedynie numer karty, data wygaśnięcia oraz kod CVV. Powodem na dość niską popularność tej metody, mogą być dwie rzeczy, 
  strach przed podawaniem danych karty, żeby nie zostały skradzione. Inną alternatywą czemu, tak mało klientów sklepów internetowych decyduję się na inne metody, 
  jest brak karty podczas składania zamówienia. Pierwsza z nich wydaję się być bardziej prawdopodobna, mało kto kupując 
  produkty przez internet ma akurat przy sobie kartę płatniczą, by odczytać z niej niezbędne liczby. 
  Zdecydowanie łatwiej jest zalogować się do banku i wykonać przelew czy to tradycyjny czy szybki.\\
  Dziwić może obecność płatności mobilnych na ostatnim miejscu z zaledwie 35\% wykorzystania. Jest to sposób bardzo wygodny, zważywszy na fakt iż mało kto 
  nie posiada przy sobie telefonu. Potrzebna jest jeszcze tylko aplikacja banku i można dokonywać dowolnych płatności. W przypadku BLIK-u generowany jest sześcio 
  cyfrowy ciąg liczb, który wystarczy wpisać w odpowienie pole, zatwierdzić płatność w aplikacji i gotowe.
}
\subsection{Bazy danych jako środek przechowywania danych}
{
  "Baza danych to zorganizowany zbiór ustrukturyzowanych informacji, czyli danych, zwykle przechowywany w systemie komputerowym w formie elektronicznej. 
  Bazą danych steruje zwykle system zarządzania bazami danych (DBMS). 
  Dane i system DBMS oraz powiązane z nimi aplikacje razem tworzą system bazodanowy, często nazywany w skrócie bazą danych.". \cite{oracle-db}
  Innymi słowy jest to kontener na dane, w dowolnej postaci, mogą to być liczby, ciągi znaków, a nawet zdjęcia czy filmy. Dane te nie są, najczęściej, przetrzymywane
  lokalnie na komputerach, tylko na serwerach czy w chmurze. \\
  Dlaczego więc korzysta się z baz danych a nie np. z arkuszy kalkulacyjnych?\\
  Odpowiedź jest bardzo prosta, arkusze kalkulacyjne nie zostały stworzone do pracy z ogromną ilością danych, przy jednoczesnym dostępie, nawet kilkuset lub więcej użytkowników.
  Są wręcz idealne do pracy z mniejszą ilościa danych dla jednego lub małej grupy użytkowników, którzy nie potrzebują wielu skomplikowanych funkcji do manipulacji danymi.
  Bazy danych z koleji, przeznaczone są do pracy z ogromnymi ilościami informacji, umożliwiając dodatkowo jednoczesny dostęp do nich, wielu użytkownikom na raz. 
  Co więcej praca z bazami danych, charakteryzuje się wysokim bezpieczeństwem i szybkością wykonywanych operacji. Operacje na danych, tworzenie zapytań odbywa się 
  za pomocą logiki i języka o wysokim stopniu złożoności. Jako przykład może posłużyć, zdecydowanie najpopularniejszy z nich, czyli język zapytań SQL.\\
  Bazy danych dzielimy, między innymi na:
  \begin{itemize}
    \item relacyjne
    \item hurtownie danych
    \item NoSQL
    \item chmurowe
  \end{itemize}
  Relacyjne bazy danych zyskały ogromną popularność w latach 80\cite{oracle-db}. Dane zorganizowane są w tabelach, składające się z wierszy i kolumn. Po dziś dzień stanowią 
  jedne z najpopularniejszych baz danych, jak nie najpopularniejsze, dostępne na rynku. Przykładami relacyjnych baz danych są: MySQL i Microsoft SQL Server.\\
  Hurtownie danych, inaczej centralne repozytorium danych, to typ bazy danych wykorzystywany do wykonywania zapytań i analizy. 
  Ma on umożliwić i wspierać działania z zakresu analizy biznesowej, w szczególności analityki. Często operuje na danych historycznych, pochadzących z wielu źródeł. 
  Jej umiejętności analityczne pozwalają przedsiębiorstwom cenne dane biznesowe, które ułatwiają podejmowanie decyzji.\cite{oracle-warehouse}\\
  Baza danych NoSQL, inaczej nierelacyjna, cechuje się przechowywaniem danych w nieuporządkowany i częściowo uporządkowanych oraz manipulowanie nimi. Od relacyjnych baz
  różni je przede wszystkim to, że w relacyjnych mają jasną strukturę organizacji danych. W nierelacyjnych dane, pochodzące z tej samej kolekcji, mogą posiadać 
  kompletnie różne atrybuty. Szerzej na temat baz NoSQL szerzej zostanie omówione w następnym rodziale.\\
  Ostatnim typem bardzo popularnym na rynku baz danych są bazy chmurowe. Bazy te charakteryzują się przede wszystkim tym, że dane przetrzymywane są na prywatnej,
  publicznej lub hybrydowej platformie przetwarzania danych w chmurze. 
  Najpopularniejszymi bazami w chmurze są Microsoft Azure SQL Database, Oracle Database, Google Cloud SQL oraz Amazon Relational Database Service.
}
\subsection{Nierelacyjne bazy danych}
{

}
\subsection{Techniki tworzenia aplikacji}
{

}
\subsection{REST API}
{

}
\subsection{Projektowanie systemu informatycznego}
{

}
}

\section{Narzędzia i technologie wybrane do realizacji projektu}
{
\subsection{Node.js}
{

}
\subsection{React.js}
{

}
\subsection{TypeScript}
{

}
\subsection{JavaScript}
{

}
\subsection{Wykorzystane najistotniejsze biblioteki}
{

}
\subsection{MongoDB}
{

}
}

\section{Proces tworzenia aplikacji}
{

}

\section{Podsumowanie}
{

}

\section{Możliwości dalszego rozwoju aplikacji}
{

}

\section{Bibliografia}
{

}

\section{Spis rysunków}
{

}

\begin{thebibliography}{0}
  \bibitem{l2short} T. Oetiker, H. Partl, I. Hyna, E. Schlegl.
    \textsl{Nie za krótkie wprowadzenie do systemu \LaTeX2e}, 2007, dostępny
    online.
  \bibitem{gemius-report} Raport dla Gemius Polska "E-commerce w Polsce 2020"
    \url{https://www.gemius.pl/wszystkie-artykuly-aktualnosci/e-commerce-w-polsce-2020.html}
  \bibitem{oracle-db} Co to jest baza danych? Oracle Polska
    \url{https://www.oracle.com/pl/database/what-is-database/}
  \bibitem{oracle-warehouse}
   \url{https://www.oracle.com/pl/database/what-is-a-data-warehouse/}
\end{thebibliography}

\end{sloppypar}
\end{document}
