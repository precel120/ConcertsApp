\documentclass[12pt]{article}
\usepackage[utf8]{inputenc}
\usepackage{color}
\usepackage{polski}
\usepackage{graphicx}
\usepackage{pdfpages}
\usepackage{indentfirst}
\usepackage{geometry}
\usepackage{setspace}

\setstretch{1.5}

 \geometry{
 a4paper,
 left=30mm,
 top=25mm,
 right=10mm,
 bottom=20mm
 }

\begin{document}
\begin{sloppypar}
\includepdf{strona_tytulowa.pdf}

\tableofcontents
\newpage

\section{Wprowadzenie}
{
  W dobie trwającego rozwoju technologicznego, każdy aspekt życia codziennego jest usprawniany i przenoszony
  do internetu. Nieinaczej jest z promowaniem i sprzedażą biletów na wydarzenie kulturowe pokroju:
  spektakli w teatrze, koncertów, festiwali. Aplikacja ConcertsApp ma służyć dokładnie temu,
  ma zminimalizować czas, jaki należałoby kiedyś przeznaczyć na zdobycie biletów. Czy też na uzyskanie
  informacji o wydarzeniach z interesującej konsumenta dziedziny, lub odbywającę się w pobliżu.
\subsection{Problematyka}
{
  
}
\subsection{Cel i założenia pracy}
{
  Celem niniejszej pracy dyplomowej jest stworzenie aplikacji webowej, która umożliwi w łatwy i szybki
  sposób na promowanie wydarzeń kulturowych i zakup biletów na nie. Zakup odbywać się będzie za pomocą
  płatności online, jedynie podając dane karty kredytowej lub debetowej.\par
  Zakres pracy obejmuję zaprojektowanie i implementację aplikacji klienckiej i serwerowej, przy
  wykorzystaniu, najnowszych i bardzo popularnych na rynku pracy, technologii. Do strony klienckiej
  został wykorzystany \textit{React}, z koleji do stworzenia serwera użyto \textit{Node.JS}.
}
\subsection{Struktura pracy inżynierskiej}
{
  Pierwszy rozdział ma posłużyć jako wprowadzenie do problemu podjętego w niniejszej pracy dyplomowej.
  Następny przedstawia aspekty takie jak: konkurencyjne rozwiązania do tworzonej aplikacji,
  analiza najpopularniejszych metod płatności online, opowiada o bazach danych oraz przybliża
  metody tworzenie aplikacji webowych.
  Trzeci rozdział przedstawia wykorzystane technologie do implementacji aplikacji oraz uzasadnia dlaczego
  akurat ona została wybrana a nie inna.
}
}

\section{Część teoretyczna}
{
\subsection{Analiza istniejących portali do promowania i dystrybucji biletów na wydarzenia kulturowe}
{
  Najpopularniejszymi portalami zbliżonymi do tworzonej aplikacji są zdecydowanie:
  \begin{itemize}
    \item Eventim
    \item GoingApp
  \end{itemize}
  Szczególnie ten drugi jest wyjątkowo popularny.z
}
\subsection{Płatności internetowe i ich bezpieczeństwo}
{

}
\subsection{Analiza najpopularniejszych internetowych metod płatności}
{

}
\subsection{Bazy danych jako środek przechowywania danych}
{

}
\subsection{Nierelacyjne bazy danych}
{

}
\subsection{Techniki tworzenia aplikacji}
{

}
\subsection{REST API}
{

}
\subsection{Projektowanie systemu informatycznego}
{

}
}

\section{Narzędzia i technologie wybrane do realizacji projektu}
{
\subsection{Node.js}
{

}
\subsection{React.js}
{

}
\subsection{TypeScript}
{

}
\subsection{JavaScript}
{

}
\subsection{Wykorzystane najistotniejsze bibliotekih}
{

}
\subsection{MongoDB}
{

}
}

\section{Proces tworzenia aplikacji}
{

}

\section{Podsumowanie}
{

}

\section{Możliwości dalszego rozwoju aplikacji}
{

}

\section{Bibliografia}
{

}

\section{Spis rysunków}
{

}



\begin{thebibliography}{0}
  \bibitem{l2short} T. Oetiker, H. Partl, I. Hyna, E. Schlegl.
    \textsl{Nie za krótkie wprowadzenie do systemu \LaTeX2e}, 2007, dostępny
    online.
\end{thebibliography}

\end{sloppypar}
\end{document}
