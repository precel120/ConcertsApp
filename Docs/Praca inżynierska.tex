\documentclass[12pt]{article}
\usepackage[utf8]{inputenc}
\usepackage{color}
\usepackage{polski}
\usepackage{graphicx}
\usepackage{pdfpages}
\usepackage{indentfirst}
\usepackage{geometry}
\usepackage{setspace}

\setstretch{1.5}

 \geometry{
 a4paper,
 left=30mm,
 top=25mm,
 right=10mm,
 bottom=20mm
 }

\begin{document}
\begin{sloppypar}
\includepdf{strona_tytulowa.pdf}

\tableofcontents
\newpage

\section{Wprowadzenie}
{
\subsection{Problematyka}
{

}
\subsection{Cel i założenia pracy}
{

}
\subsection{Struktura pracy inżynierskiej}
{

}
}

\section{Część teoretyczna}
{
\subsection{Analiza istniejących portali do promowania i dystrybucji biletów na wydarzenia kulturowe}
{

}
\subsection{Płatności internetowe i ich bezpieczeństwo}
{

}
\subsection{Analiza najpopularniejszych internetowych metod płatności}
{

}
\subsection{Bazy danych jako środek przechowywania danych}
{

}
\subsection{Nierelacyjne bazy danych}
{

}
\subsection{Techniki tworzenia aplikacji}
{

}
\subsection{REST API}
{

}
\subsection{Projektowanie systemu informatycznego}
{

}
}

\section{Narzędzia i technologie wybrane do realizacji projektu}
{
\subsection{Node.js}
{

}
\subsection{React.js}
{

}
\subsection{TypeScript}
{

}
\subsection{JavaScript}
{

}
\subsection{Wykorzystane najistotniejsze bibliotekih}
{

}
\subsection{MongoDB}
{

}
}

\section{Proces tworzenia aplikacji}
{

}

\section{Podsumowanie}
{

}

\section{Możliwości dalszego rozwoju aplikacji}
{

}

\section{Bibliografia}
{

}

\section{Spis rysunków}
{

}



\begin{thebibliography}{0}
  \bibitem{l2short} T. Oetiker, H. Partl, I. Hyna, E. Schlegl.
    \textsl{Nie za krótkie wprowadzenie do systemu \LaTeX2e}, 2007, dostępny
    online.
\end{thebibliography}

\end{sloppypar}
\end{document}
