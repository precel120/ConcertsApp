\documentclass[12pt]{article}
\usepackage[utf8]{inputenc}
\usepackage{color}
\usepackage{polski}
\usepackage{graphicx}
\usepackage{pdfpages}
\usepackage{indentfirst}
\usepackage{geometry}
\usepackage{setspace}

\setstretch{1.5}

 \geometry{
 a4paper,
 left=30mm,
 top=25mm,
 right=10mm,
 bottom=20mm
 }

\begin{document}
\begin{sloppypar}
\includepdf{strona_tytulowa.pdf}

\tableofcontents
\newpage

\section{Wprowadzenie}
{
  W dobie trwającego rozwoju technologicznego, każdy aspekt życia codziennego jest usprawniany i przenoszony
  do internetu. Nieinaczej jest z promowaniem i sprzedażą biletów na wydarzenie kulturowe pokroju:
  spektakli w teatrze, koncertów, festiwali. Aplikacja ConcertsApp ma służyć dokładnie temu,
  ma zminimalizować czas, jaki należałoby kiedyś przeznaczyć na zdobycie biletów. Czy też na uzyskanie
  informacji o wydarzeniach z interesującej konsumenta dziedziny, lub odbywającę się w pobliżu.
\subsection{Problematyka}
{
  
}
\subsection{Cel i założenia pracy}
{
  Celem niniejszej pracy dyplomowej jest stworzenie aplikacji webowej, która umożliwi w łatwy i szybki
  sposób na promowanie wydarzeń kulturowych i zakup biletów na nie. Zakup odbywać się będzie za pomocą
  płatności online, jedynie podając dane karty kredytowej lub debetowej.\par
  Zakres pracy obejmuję zaprojektowanie i implementację aplikacji klienckiej i serwerowej, przy
  wykorzystaniu, najnowszych i bardzo popularnych na rynku pracy, technologii. Do strony klienckiej
  został wykorzystany \textit{React}, z koleji do stworzenia serwera użyto \textit{Node.JS}.
}
\subsection{Struktura pracy inżynierskiej}
{
  Pierwszy rozdział ma posłużyć jako wprowadzenie do problemu podjętego w niniejszej pracy dyplomowej.
  Następny przedstawia aspekty takie jak: konkurencyjne rozwiązania do tworzonej aplikacji,
  analiza najpopularniejszych metod płatności online, opowiada o bazach danych oraz przybliża
  metody tworzenie aplikacji webowych.
  Trzeci rozdział przedstawia wykorzystane technologie do implementacji aplikacji oraz uzasadnia dlaczego
  akurat ona została wybrana a nie inna.
  Czwarty rozdział pokazuje jak przebiegał proces twórczy, czyli projektowanie i implementacja.
}
}

\section{Część teoretyczna}
{
\subsection{Analiza istniejących portali do promowania i dystrybucji biletów na wydarzenia kulturowe}
{
  Najpopularniejszymi portalami zbliżonymi do tworzonej aplikacji są zdecydowanie:
  \begin{itemize}
    \item Eventim
    \item GoingApp
  \end{itemize}
  Eventim jest znacznie starszym portalem, co można stwierdzić chociażby, po jego szacie graficznej,
  łatwo to zauważyć porównująć do wyżej wymienionego GoingApp. 
  W swojej ofercie ma wydarzenia pokroju: koncertów, przedstawień teatralnych, oper czy baletów.\\
  Same bilety sprzedawane są w formie papierowej, przychodzą na maila, 
  lub istnieje możliwość zakupu biletu mobilnego. Kupno może odbyć się zarówno poprzez strone internetową, 
  jak również poprzez aplikację mobilną. Płatności można dokonać za pomocą przelewu tradycyjnego, karty kredytowej, szybkiego przelewu Dotpay\\
  GoingApp charakteryzuje się bardziej nowoczesną i przejrzystą szatą graficzną w porównaniu do Eventim. 
  Oferty obu portali są bardzo do siebie zbliżone, lecz tutaj można znaleźć takie wydarzenia, jak chociażby imprezy związane z filmem czy jedzeniem.
  Jednakże nie posiada on biletów na balet lub opery.\\
  Wejściówki można, podobnie jak w Eventim nabyć poprzez ich stronę internetową lub aplikację mobilną. 
  Sam bilet jest dostępny w formie dokumentu PDF, lub kodu QR, dostęp do niego mamy zarówno poprzez maila,którego otrzymujemy zaraz po zakupie, oraz aplikację.
  Użytkownik może dokonać płatności za pomocą systemów płatniczych takich jak: payU, eCard, MasterPass oraz Paymento\textregistered \\
  Aplikacja webowa tworzona na potrzeby niniejszej pracy inżynierskiej czerpie z obu portali najlepsze cechy. 
  Z Eventim czerpie różnorodność wydarzeń, z koleji z GoingApp przejrzystość interfejsu i bilety w formie, bardzo popularnego kodu QR.
  Metody płatności zostają ograniczone do kart płatniczych: debetowej i kredytowej. Podawane są numer karty, data ważności oraz kod CVV.
}
\subsection{Płatności internetowe i ich bezpieczeństwo}
{

}
\subsection{Analiza najpopularniejszych internetowych metod płatności}
{

}
\subsection{Bazy danych jako środek przechowywania danych}
{

}
\subsection{Nierelacyjne bazy danych}
{

}
\subsection{Techniki tworzenia aplikacji}
{

}
\subsection{REST API}
{

}
\subsection{Projektowanie systemu informatycznego}
{

}
}

\section{Narzędzia i technologie wybrane do realizacji projektu}
{
\subsection{Node.js}
{

}
\subsection{React.js}
{

}
\subsection{TypeScript}
{

}
\subsection{JavaScript}
{

}
\subsection{Wykorzystane najistotniejsze biblioteki}
{

}
\subsection{MongoDB}
{

}
}

\section{Proces tworzenia aplikacji}
{

}

\section{Podsumowanie}
{

}

\section{Możliwości dalszego rozwoju aplikacji}
{

}

\section{Bibliografia}
{

}

\section{Spis rysunków}
{

}

\begin{thebibliography}{0}
  \bibitem{l2short} T. Oetiker, H. Partl, I. Hyna, E. Schlegl.
    \textsl{Nie za krótkie wprowadzenie do systemu \LaTeX2e}, 2007, dostępny
    online.
\end{thebibliography}

\end{sloppypar}
\end{document}
